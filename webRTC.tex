WebRTC :- WebRTC is a set of javascript API's that allow us to establish peer to peer connection between two browsers to exchange data such as audio and video in real time.
 -- Real time Connection
 -- Lower Latency

-----------------

machine know their private IP address but not public IP address

-----------------

WebSockets and WebRTC difference
 --WebSockets:- Real Time Communication through server( transports its data through TCP )
 --WebRTC:- Real Time Communication between browsers, data never touches server( WebRTC transports its data over UDP, UDP is FAST)

 -----------------

So, if WebRTC is so fast, why use websockets at all?
 --UDP is not a reliable protocol for transferring important data
 --No build in signaling
 --UDP does not  validate data

 -----------------

So, what exactly is sent between the two clients and how is it send?

 --SDP's :- A Session Description Protocol(SDP), is an object containing information about the session connection such as the codec, address, media type, audio and video and so on.
            -- it is a text-based format that describes  the parameters and attributes of a multimedia session. SDP is used to support real-time services like video conferencing, Voip calls, and streaming  music. SDP uses protocols like RTP, RTSP and SIP.
               ( information of the machine)( give sdp or information to node server to the other machine).   

 --ICE Candidates :- An ICE candidate is a public IP address and port that could potentially be an address that receives data.   ( interactive Connectivity establishment)
                     -- when starting a WebRTC peer connection, the candidates negotiate with each other until they agree on one. WebRTC then uses that candidate's details to establish the connection.
                     ( ice server tells us the public IP address to the machine (machine request to ice server for that)).
                     ( only two people can connect)(peer-to-peer)
-----------------

STUN Server :-(Session Traversal of User Datagram Protocol through Network Address Translations).[it is a key part real-time Communication]. 
            :- is a standardized set of methods, including a Network protocol, for Traversal of network address translator(NAT) gateways in application of real time voice, video, messaging, and other interactive communications.
            :- server is used to facilates peer-to-peer(p2p) and voice over internet protocol(voip) connections.
            :- it allows clients to discover their public address , the type of NAT they are behind, and the port mapping behaviour of the NAT.
            
            ( devices ko private ip address pta rhta hai, par unha public ip address nhi pta hota)

NAT :- ( Network Address Translation):- 
            :- is a technique used to allow multiple devices on a private network to share a single public IP address when accessing the internet or other external networks.
            :- when a device on a private network sends a request to access a resource on th internet,  (the NAT router translates the private IP address of the device to the public IP address). when a response from the internet is received, the NAT router translates the public IP address back to the private IP address of the device  that made the request.
       
        Why Use NAT:-
        IP Conservation:- NAT helps conserve IP addresses by allowing multiple devices to share a single public IP address
        Security:- NAT provides a level of Security
        Cost Effective:- it reduces the need for multiple public IP addresses, which can be costly

        Limitations:-
        Scalability:- become complex and difficult to manage as the number of devices on network increases.
        Performance:- NAT can introduce Latency and affect network performance due to additional processing required to translates IP addresses.
        limited support for multicast:-  

----------------

---Types of architecture in web WebRTC
    there are three types of architectures in web WebRTC
    (i).peer-to-peer Architecture
    (ii).Mesh Architecture
    (iii).SFU( Selective Forwarding Unit) Architecture
    (iv).MCU( Multipoint Control Unit) Architecture



